%-*- coding: UTF-8 -*
% gougu.tex
% 勾股定理
%%%%%		注释:说明文件编码、说明源文件文件名、说明源文件内容
\documentclass[UTF8]{ctexart}
%%%%%		\documentclass{}文档类		ctexart:中文文档		UTF8:文件编码
\usepackage{graphicx}
\usepackage{float}
\usepackage{amsmath}
\usepackage{geometry}
\geometry{a6paper,centering,scale=0.8}
\usepackage[format=hang,font=small,textfont=it]{caption}
\usepackage[nottoc]{tocbibind}

\title{\heiti 杂谈勾股定理}
\author{\kaishu 张三}
\date{\today}
%%%%%		\title{}、\author{}、\date{}分别声明文档标题、作者、写作时间		\today是今天的日期

\bibliographystyle{plain}
%%%%%		\bibliographystyle{}声明参考文献格式
\newtheorem{thm}{定理}

%%%%%		\begin{document}及\end{document}声明文档环境——论文正文部分
\begin{document}

\maketitle
%%%%%		\maketitle实际输出论文标题
\begin{abstract}
	这是一篇关于勾股定理的小短文。
\end{abstract}
%%%%%		\begin{abstract}和\end{abstract}利用abstract环境生成文章摘要
\tableofcontents
%%%%%		\tableofcontents输出目录
\section{勾股定理在古代}
西方称勾股定理为毕达哥拉斯定理,将勾股定理的发现归功于公元前6世纪的毕达哥拉斯学派\cite{Kline}。该学派得到一个法则,可以求出可以排成直角三角形三边的三元数组。毕达哥拉斯学派没有书面著作,该定理的严格表述和证明则见于欧几里德\footnote{欧几里德,约公元前 330--275 年。}《几何原本》的命题 47:“直角三角形斜边上的正方形等于两只脚边上的两个正方形之和。”正面是用面积做的。
%%%%%		忽略各行开头的所有空格
%%%%%		汉字后空格会被忽略,其他符号后则保留
%%%%%		一个空格足矣,多个空格无用
%%%%%		\footnote{}	脚注命令

%%%%%		需使用空行分隔,以实现分段
%%%%%		一个空行足矣,多个空行无用
我国《周髀算经》载商高(约公元前 12 世纪)答周公问:
\begin{quote}
	\zihao{-5}\kaishu 勾广三,股修四,径隅五。
%%%%%		\zihao{}改变字号,相对减小或增大
\end{quote}
%%%%%		\begin{quote和\end{quote}使用quote环境,突出引用的部分		单独分行、增加缩进和上下间距排印
又载陈子(约公元前7-6世纪)答荣方问:
\begin{quote}
	\zihao{-5}\kaishu 若求邪至日者,以日下为勾,日高为股,勾股各自乘,并而开方除之,得邪至日。
\end{quote}
都较古希腊更早。后者已经明确道出勾股定理的一般形式。图 \ref{fig:xiantu}是我国古代对勾股定理的一种证明\cite{quanjing}。
\begin{figure}[ht]
	\centering
	\includegraphics[scale=0.6]{xiantu.pdf}
	\caption{宋赵爽在《周髀算经》注中作的弦图(仿制),该图给出了勾股定理的极具对称美的证明。}
	\label{fig:xiantu}
\end{figure}

\section{勾股定理的现代形式}

勾股定理可以用现代语言表述如下:
\begin{thm}[勾股定理]
直角三角形斜边的平方等于两腰的平方和。
可以用符号语言表述为:设直角三角形 ABC,其中$\angel C = $90^\circ$,则有
\begin{equation}
\label{eq:gougu}
AB^2 = BC^2 + AC^2
\end{equation}
\end{thm}
满足式的整数称为\emph{勾股数}。第 1 节所说毕达哥拉斯学派得到的三元数组就是勾股数。下表列出了一些较小的勾股数:
%%%%%		\emph{}改变字体形状,表示强调内容
\begin{table}[H]
\begin{tabular}{|rrr|}
\hline
直角边 $a$ & 直角边 $b$ & 斜边 $c$ \\
\hline
	3 & 4 & 6 \\
	5 & 12 & 13 \\
\hline
\end{tabular}
\qquad
($a^2 + b^2 = c^2$)
\end{table}

\nocite{Shiye}
\bibligraphy{math}
%%%%%		从文献档案库math获取文献信息,打印参考文献列表

\end{document}
