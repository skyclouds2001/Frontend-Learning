\documentclass{article}
\usepackage[fleqn]{amsmath}
\usepackage{amssymb,graphicx,color,graphicx,slashed, microtype, parskip, enumitem, extarrows, needspace}

\usepackage[top=1.5cm, bottom=1.5cm, right=6cm, left=1.5cm, heightrounded, marginparwidth=5cm, marginparsep=0.5cm]{geometry}

\hbadness = 10000
\hfuzz=100pt 
    
\usepackage{marginnote}
\renewcommand*{\marginfont}{\footnotesize}

\usepackage{hyperref}
\hypersetup{colorlinks=true, urlcolor=NavyBlue, bookmarksdepth=3}

\makeatletter\newcommand{\@minipagerestore}{\setlength{\parskip}{\medskipamount}}\makeatother

% =============== Index ===========================

\usepackage[nonewpage]{imakeidx}
\makeindex

% =============== Color Definitions ===============
    
\usepackage[svgnames]{xcolor}
\colorlet{ColorTitle}{Black}
\colorlet{ColorSectionName}{Black}
\colorlet{ColorBoxFG}{Gray}
\colorlet{ColorBoxText}{Black}
\colorlet{ColorBoxBG}{White}


% =============== Title Style ===============
    
\usepackage{titling} % Allows custom title configuration
    
\newcommand{\HorRule}{\color{ColorTitle}\rule{\linewidth}{1pt}} % Defines the gold horizontal rule around the title
    
\pretitle{
    \vspace{-50pt} % Move the entire title section up
    \HorRule\vspace{9pt} % Horizontal rule before the title
    \fontsize{27}{36}\usefont{OT1}{phv}{b}{n}\selectfont
    \color{ColorTitle} % Text colour for the title and author(s)
}
    
\posttitle{\par\vskip 15pt} % Whitespace under the title
    
\preauthor{\fontsize{17}{0}\usefont{OT1}{phv}{m}{n}\selectfont\color{ColorTitle}} % Anything that will appear before \author is printed
    
\postauthor{\par\HorRule}

\newcommand{\COURSENAME}{\href{http://phyw.people.ust.hk/teaching/PHYS2022-2015/}{\textcolor{black}{PHYS 2022}}}
\newcommand{\YW}{\href{http://phyw.people.ust.hk/}{\textcolor{black}{Yi Wang}}}
\newcommand{\PHYS}{\href{http://physics.ust.hk}{\textcolor{black}{Department of Physics}}}
\newcommand{\HKUST}{\href{http://www.ust.hk/}{\textcolor{black}{HKUST}}}
\author{\COURSENAME, \YW, \PHYS, \HKUST}

\date{}

% =============== Section Name Style ===============
    
\usepackage{titlesec}
    
\titleformat{\section}
    {\fontsize{15}{20}\usefont{OT1}{phv}{b}{n}\color{ColorSectionName}}
    {\thesection}{1em}{}
    %[{\vspace{0.2cm}\titlerule[0.8pt]}]
    
\titleformat{\subsection}
    {\fontsize{14}{20}\usefont{OT1}{phv}{m}{n}\color{ColorSectionName}}
    {\thesubsection}{1em}{}
    
\titleformat{\subsubsection}
    {\fontsize{12}{20}\usefont{OT1}{phv}{m}{n}\color{ColorSectionName}}
    {}{0em}{}
      
\setcounter{secnumdepth}{4}
        
% =============== Box Style ===============
    
\usepackage[most]{tcolorbox}
    
\newtcolorbox{tbox}[1]{
    colback=ColorBoxBG, colframe=ColorBoxFG, coltext=ColorBoxText,
    sharp corners, enhanced, breakable, parbox=false,
    before skip=1em, after skip=1em,
    title={#1}, fonttitle=\usefont{OT1}{phv}{b}{n}, 
    attach boxed title to top left={yshift=-0.1mm}, boxed title style={sharp corners, colback=ColorBoxFG, left=0.405cm},
    rightrule=-1pt,toprule=-1pt, bottomrule=-1pt
}

\newtcolorbox{mtbox}[1]{
    colback=ColorBoxBG, colframe=ColorBoxFG, coltext=ColorBoxText,
    sharp corners, enhanced, breakable, parbox=false,
    before skip=1em, after skip=1em,
    title={#1}, fonttitle=\usefont{OT1}{phv}{b}{n},
    attach boxed title to top left={yshift=-0.1mm}, boxed title style={sharp corners, colback=ColorBoxFG, left=0.15cm},
    rightrule=-1pt,toprule=-1pt, bottomrule=-1pt, 
    left=0.5em
}

% =============== tikz has to be loaded after xcolor
\usepackage{tikz}

\newcommand*\enumlabel[1]{\tikz[baseline=(char.base)]{
			\node[shape=rectangle,inner sep=2pt,fill=ColorBoxFG] (char) 
			{\fontsize{7}{20}\usefont{OT1}{phv}{b}{n}{\textcolor{ColorBoxBG}{#1}}};}}

% =============== Useful shortcuts ===============

\newcommand\wref[1]{{\hypersetup{linkcolor=white}\ref{#1}}}  

\newcommand{\textbox}[2]{
    \begin{tbox}{#1}
        #2
    \end{tbox}
}

\newcommand{\mtextbox}[2]{\marginnote{
    \begin{mtbox}{#1}
        #2
    \end{mtbox}}
}

\newcommand{\mnewline}{\vspace{0.5em}\newline}

\newcommand{\titem}[1]{
    \begin{itemize}[label=\color{ColorBoxFG}$\blacktriangleright$, leftmargin=0mm, labelsep=0.27cm, topsep=0.5em
        %, itemsep=1ex
        ]
        #1
    \end{itemize}
}

\newcommand{\mtitem}[1]{
    \begin{itemize}[label={\color{ColorBoxFG}$\blacktriangleright$}, leftmargin=0mm, labelsep=1mm, topsep=0.5em
        %, itemsep=1ex
        ]
        #1
    \end{itemize}
}

\newcommand{\itembox}[3]{
    \begin{tbox}{#1}
        #2
        \titem{#3}
    \end{tbox}
}

\newcommand{\mitembox}[3]{
    \marginnote{
    \begin{mtbox}{#1}
        #2
        \mtitem{#3}
	\end{mtbox}
    }
}

\newcommand{\tenum}[1]{
    \begin{enumerate}[label=\protect\enumlabel{\arabic*}, leftmargin=0mm, labelsep=0.265cm, topsep=0.5em
        %, itemsep=1ex
        ]
        #1
    \end{enumerate}
}

\newcommand{\enumbox}[3]{
    \begin{tbox}{#1}
        #2
        \tenum{#3}
    \end{tbox}
}

\newcommand{\twocol}[5]{
    \begin{minipage}[t][][b]
        {#1\textwidth}
        #4        
    \end{minipage}
    \hspace{#2\textwidth}
    \begin{minipage}[t][][b]
        {#3\textwidth}
        #5
    \end{minipage}
}

\newcommand{\cg}[2]{
    \begin{center}
        \includegraphics[width=#1\textwidth]{#2}
    \end{center}
}

\newcommand{\tbar}{
    ~\newline
    {\color{ColorBoxFG}
    \hbox to 0.15\textwidth{\leaders\hbox to 5pt{\hss  \hss}\hfil} 
    \hbox to 0.7\textwidth{\leaders\hbox to 5pt{\hss . \hss}\hfil}}
    \mnewline
}

% =============== Filter unwanted warnings
\usepackage{silence}
\WarningsOff[tcolorbox]
\hbadness=1000000

\title{Part 5. Atoms}

\begin{document}

\maketitle

\textbox{Feynman's question}{
    As the opening of his lectures, Feynman asked the following question:
    \begin{quote}
        If, in some cataclysm, all of scientific knowledge were to be destroyed, and only one sentence passed on to the next generations of creatures, what statement would contain the most information in the fewest words? 
    \end{quote}
    He gave an answer himself:
    \begin{quote}
        Matter is made of atoms.
    \end{quote}
    This statement is not absolutely right. For example, E\&M waves may be considered a form of matter which is not made of atoms (though made of quanta). Dark matter is not made of conventional atoms either, and dark energy does not look like atoms by all means. Nevertheless, our familiar matter world is indeed made of atoms. 
    \marginnote{
        With modern technology, \ref{item:atom-chem}, \ref{item:atom-size} and \ref{item:atom-evid} seems trivial. Because scanning tunneling microscopes can directly see and manipulate atoms. However, back to 150$\sim$200 years ago, how these features were known from scientific methods? Strictly speaking, they are not part of modern physics. But as it is not completely covered in general physics, I decide to include it here.
    }
    And I agree that this is the message that we should pass on. 
    \tcblower
    How do we know, why do we care, and what are the consequences that the world is made of atoms? This will be the focus of this part. More explicitly, we will address:
    \tenum{
        \item How the atomic theory arised in chemistry?\label{item:atom-chem}
        \item How do we know the size of an atom?\label{item:atom-size}
        \item Can we find direct evidences for the existence of atoms?\label{item:atom-evid}
        \item How can an atom be stable? -- How does quantum mechanics save the world?\label{item:atom-quantum}
        \item Where do the chemical natures of atoms arise?\label{item:atom-chem-nature}
    }
    \mtextbox{Oil film method}{
        Independently, the size of atoms (molecules) can also be determined by oil film method. Franklin (1757) noted that oil can spread on a huge area of water. The thin film of oil upon water can be as thin as a single layer of molecule. However, such huge area is hard to measure. Is it possible to make the amount of oil smaller?
        \tcblower
        Langmuir (1917) used alcohol to dissolve oleic acad. Drip one drop of such solution to water. Alcohol is dissolved by water and oleic acad spread on the surface of water with an area measurable in a lab.
    }
}

\end{document}